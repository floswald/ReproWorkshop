\documentclass[12pt]{article}
\usepackage{setspace}
\usepackage{geometry}
\usepackage{adjustbox}
\geometry{verbose,lmargin=2cm,rmargin=2cm,bmargin=2.5cm,tmargin=2cm}

\usepackage{booktabs}
\usepackage{natbib}
\usepackage{hyperref}
\hypersetup{colorlinks=true,linkcolor=blue,urlcolor=blue}



\newcommand{\rootdir}{..}  % one dir up
\newcommand{\plots}{\rootdir/output/plots}
\newcommand{\tables}{\rootdir/output/tables}


%% TITLE PAGE
\begin{document}
\onehalfspacing
    
    
\title{A Reproducible Benchmark of Fixed Effects Estimation}



\author{Florian Oswald\thanks{RES Data Editor. You can find the source code generating the entire workshop at \url{https://github.com/floswald/ReproData.jl}}}
\date{\today}


\maketitle
\begin{abstract}
We illustrate a workflow which tries to address several pitfalls when creating a reproducible research project. We focus on preserving raw data, documenting data sources, creating a folder structure, writing stata and R code in a way which helps to preserve the package version environment, outputting results to disk and referencing them in a final output document. As a by-product, we report timings of a typical two-way fixed effects estimation exercise on a large dataset.    
\end{abstract}

\section{Introduction}

One could be forgiven to think that reproducibility is as simple as following a few simple steps:

\begin{enumerate}
\item Preserve raw data
\item Document data origin
\item Preserve code and document how to use it
\item Run everything again before submitting the package.
\end{enumerate}

While this is a good start, this list if far from exhaustive, and a more complete version is available under \url{https://datacodestandard.org}. Whatever the list, however, the devil is in the details, and \emph{in practice} achieving reproducibility is far from trivial. We want to use this fictitious research project to illustrate one potential strategy when setting up code, and data, and a few associated pitfalls. 

\section{Computational Task}

In this paper, we want to estimate the following linear regression with two fixed effects:

\begin{equation}
y_{ij} = \beta X_{ij} + \alpha_i + \gamma_j + u_{ij} \label{eq:1}
\end{equation}
where $X_{ij}$ is a matrix which stacks the 1 by 7 vectors $[ x_{ij1}, \dots, x_{ij7}]$. The indices $(i,j)$ group observations along two ad-hoc dimensions: imagine person and time, or worker and firm specific effects. Those $\alpha_i,\gamma_j$ are unobservable.

We generated the data such that the first $x$ is a function of both fixed effects, $x_{it1} = g(\alpha_i, \gamma_t)$, the second a function only of $\gamma_t$, $x_{it2} = h(\gamma_t)$, and we set the true values for coefficients to $\beta = [ 3,3,1,1,1,1,1]$. Now let me show you the first result in table \ref{tab:1}. Observe that models (1) and (2) exhibit bias, and only after we account for both fixed effects, we get the correct results. Overall this seems to work.\footnote{The interested reader may consult the data generating process \href{https://github.com/floswald/ReproData.jl/blob/main/src/ReproData.jl}{here}.}

\begin{table}
\centering
{
\def\sym#1{\ifmmode^{#1}\else\(^{#1}\)\fi}
\begin{tabular}{l*{4}{c}}
\toprule
                    &\multicolumn{1}{c}{(1)}&\multicolumn{1}{c}{(2)}&\multicolumn{1}{c}{(3)}&\multicolumn{1}{c}{(4)}\\
                    &\multicolumn{1}{c}{y}&\multicolumn{1}{c}{y}&\multicolumn{1}{c}{y}&\multicolumn{1}{c}{y}\\
\midrule
x1                  &       3.982\sym{***}&       3.494\sym{***}&       3.001\sym{***}&       2.998\sym{***}\\
                    &   (0.00108)         &   (0.00553)         &   (0.00778)         &   (0.00318)         \\
\addlinespace
x2                  &       3.019\sym{***}&       3.497\sym{***}&       3.003\sym{***}&       3.001\sym{***}\\
                    &   (0.00151)         &   (0.00553)         &   (0.00779)         &   (0.00318)         \\
\addlinespace
x3                  &                     &                     &                     &       1.000\sym{***}\\
                    &                     &                     &                     &  (0.000318)         \\
\addlinespace
x4                  &                     &                     &                     &       1.000\sym{***}\\
                    &                     &                     &                     &  (0.000318)         \\
\addlinespace
x5                  &                     &                     &                     &       1.001\sym{***}\\
                    &                     &                     &                     &  (0.000318)         \\
\addlinespace
x6                  &                     &                     &                     &       1.000\sym{***}\\
                    &                     &                     &                     &  (0.000318)         \\
\addlinespace
x7                  &                     &                     &                     &       1.000\sym{***}\\
                    &                     &                     &                     &  (0.000318)         \\
\addlinespace
Constant            &     0.00114         &    0.000955         &   -0.000234         &    -0.00164\sym{***}\\
                    &  (0.000775)         &  (0.000775)         &  (0.000775)         &  (0.000316)         \\
\midrule
FE 1                &          No         &         Yes         &         Yes         &         Yes         \\
FE 2                &                     &                     &                     &                     \\
Observations        &    10000000         &    10000000         &    10000000         &    10000000         \\
\bottomrule
\multicolumn{5}{l}{\footnotesize Standard errors in parentheses}\\
\multicolumn{5}{l}{\footnotesize \sym{*} \(p<0.10\), \sym{**} \(p<0.05\), \sym{***} \(p<0.01\)}\\
\end{tabular}
}

\caption{This is done with stata. I couldn't figure out why the FE2 row does not display a "yes" in columns 3 and 4. My bad, sorry!\label{tab:1}}
\end{table}
\end{document}
